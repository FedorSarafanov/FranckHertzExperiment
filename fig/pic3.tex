\documentclass[tikz]{standalone}

\usepackage[english,russian]{babel}
\usepackage[defaultsans]{droidsans}
\renewcommand*\familydefault{\sfdefault} %% Only if the base font of the document is to be typewriter style
\usepackage[T2A,T1]{fontenc}
\usepackage[utf8]{inputenc}
\input{tikz.tex} 
\usepackage
    {
        tikz,
        pgfplots,
        verbatim,
    }
\usetikzlibrary
    {
        arrows,
        patterns,
        angles,
        quotes,
        calc, 
        3d,  
        backgrounds, 
        positioning
    }
\usetikzlibrary{fadings,through}
\tikzset
    {
        media/.style={font={\footnotesize\sffamily}},
        wave/.style={
            decorate,decoration={snake,post length=1.4mm,amplitude=2mm,
                segment length=1mm},thick},   
    }

\definecolor{c0}{HTML}{1f77b4}
\definecolor{c1}{HTML}{ff7f0e}
\definecolor{c2}{HTML}{2ca02c}
\definecolor{c3}{HTML}{d62728}

\usepackage[outline]{contour}
\contourlength{2.2pt}
\begin{document}
\begin{tikzpicture}
\xdef\SIZE{4}
% \xdef\setka{1}
\input{setka}

\draw[very thick] (0,0) node[left] {К} -- ++(4,0);
\draw[very thick] (0,2) node[left] {С} -- ++(4,0);
\draw[very thick] (0,4) node[left] {А} -- ++(4,0);

\foreach \y in {0,0.33, ..., 1.66}{
    \draw [densely dashed] (0,\y) -- ++ (4,0);
}

\foreach \y in {2,2.66,...,4}{
    \draw [densely dashed] (0,\y) -- ++ (4,0);
}

\draw[pattern=north west lines,draw=none, pattern color=black] (0,1.5) rectangle (4,2);
\draw[dashed] (0,1.5) -- ++(4,0);

\foreach \x in {1,2,3}{
    \draw [<->,>=latex] (\x,0) -- ++ (0,4);
}

\foreach \x in {0.5,1.5,...,3.5}{
    \draw [<-,>=latex] (\x,0) -- ++ (0,2);
}

% \draw (2,1.7) node[fill=white] {$\Omega$};

\node at (2,1.7) {\contour{white}{$\Omega$}};
\begin{scope}[xshift=4cm,yshift=2cm]
 \lineannt[-0.8]{90}{-0.5}{$\lambda$}
\end{scope}
\node at (4.6,1.1) {$\varphi_1$};
\node at (5.1,2.4) {$\varphi_1+\lambda\frac{d\varphi}{dn}$};
% \draw[pattern=north west lines,draw=none, pattern color=black] (2,0) rectangle (4,4);

% \draw[fill=magenta!15, draw=none] (1,5) -- (1,3) -- (1.5,2) -- (3.5,2) -- (4,3) -- (4,5) -- cycle;

% \draw[line width=4pt] (1,5) -- (1,3) -- (1.5,2);
% \draw[line width=4pt] (3.5,2) -- (4,3) -- (4,5);

% \draw[line width=4pt] (2.5,5) -- (2.5,2);
% \draw[line width=2pt, black] (2.5,2) -- (3.5,2);

% \draw (2.5,5) node[above] {Коаксиальный кабель};

% \draw (3,2) node[] (t1) {};
% \draw (5,2.5) node[] (t2) {Зонд};

% \path[->, very thick] (t2) edge [out=-90, in=-90] (t1);

% \node (s1) at (0,0.25){};
% \node (s2) at (4.5,0.6){};
% \draw[fill=black] (0,0)  rectangle (5,0.5);
% \draw[fill=black!60] (0.5,0.5) rectangle ++(4,0.2) node (t2) {};

% \draw (2.5,0.25) node[white] (ss1) {ВТСП-пленка} node[left, xshift=-3em] (ss11) {};
% \draw (2.5,+1) node[] (ss2) {Тефлоновая пленка} node[right, xshift=4.5em] (ss22) {};
% % \draw[draw=white] (3,0) node[above,white]  {$\vec{j}_{pov}$};

% % \path[->, very thick] (ss11) edge [out=180, in=-180] (s1);
% % \path[->, very thick] (ss22) edge [out=0, in=0] (s2);
\end{tikzpicture} 
\end{document}