\documentclass[border=1pt]{standalone}
\usepackage[europeanresistors,americaninductors]{circuitikz}
\usetikzlibrary
    {
        arrows,
        patterns,
        angles,
        quotes,
        calc, 
        3d,  
        backgrounds, 
        positioning
    }
\begin{document}
\begin{circuitikz}

\ctikzset {label/align = straight }
\ctikzset{bipoles/ammeter/text rotate/.initial=0,% <=new key
rotation/.style={bipoles/ammeter/text rotate=#1},% style for ease introduction in code
}


% %%%%%%%%% ЛАМПА %%%%%%%%%%%%%%%%%
\begin{scope}
    \clip(0,0) circle (0.5cm);
    \draw (0,-0.5) circle (0.25cm);
    \draw[dashed] (-0.5,0) node[left] {} -- (0.5,0) coordinate (S);
    \draw (0,0.5) coordinate (A) -- ++(0,-0.2) coordinate (1232);
    \draw (1232)++(-0.2,0) -- ++(0.4,0);
    \coordinate (K1) at ($(0,-0.5)+({180-(90-75.5224878)}:0.25)$);
    \coordinate (K2) at ($(0,-0.5)+({(90-75.5224878)}:0.25)$);
\end{scope} 
\draw (K2) -- (0.25,-0.5) coordinate (K2);
\draw (K1) -- (-0.25,-0.5) coordinate (K1);
\draw (0,0) circle (0.5cm);
% %%%%%%%%%%%%%%%%%%%%%%%%%%%%%%%%%

\draw (A) -- ++(0,1.5) 
    -- node[midway] {} coordinate (mua) ++(2,0) coordinate (vz1)
    -- node[midway] {} coordinate (v1) ++(0,-2) coordinate (vz2);
\draw (vz2) to ++(-1.5,0);

\draw (K2) -- ++(0,-1.5) coordinate (ky);
\draw (K1) -- ++(0,-1.5) coordinate (ky2);
\draw (ky) -- (2,-2) coordinate (vy);

\draw (vy) -- node[midway] {} coordinate (v2) ++(0,2) coordinate (vzvy);

\draw (4,0) coordinate (Rz1) to[potentiometer,n=Rz] ++(0,2) coordinate (Rz2);

\draw (4,-2) coordinate (Ry1) to[potentiometer,n=Ry] ++(0,1.7) coordinate (Ry2);
 % \draw (0,0) to[potentiometer,n=Rz] ++(2,0)
  % (Rz.wiper) to[short,-o] ++(2,0);

% \draw ()

\draw[thick, fill=white] (v1) circle (0.4cm) node {$V_z$};
\draw[thick, fill=white] (v2) circle (0.4cm) node {$V_y$};
\draw[thick, fill=white] (mua) circle (0.4cm) node {$\mu A$};

\draw (vz2) to[short,*-] (Rz1);
\draw (vz1) to[short, *-] ++(0,0) -| (Rz.wiper);
\draw (3,0) to[short, *-] ++(0,0) |- (Ry.wiper);

\draw (2,-2) to[short, *-] ++(2,0);

\draw (4,2) -- ++ (1,0) -- ++ (0,-0.6) coordinate (u1);
\draw (4,0) -- ++ (1,0) -- ++ (0,+0.6) coordinate (u2);
\draw[fill=white] (u1) circle (2pt) node[right] {$-$};
\draw[fill=white] (u2) circle (2pt) node[right] {$+$};

\draw (4,{-2+1.7}) -- ++ (1,0) -- ++ (0,-0.5) coordinate (u1);
\draw (4,-2) -- ++ (1,0) -- ++ (0,+0.5) coordinate (u2);
\draw[fill=white] (u1) circle (2pt) node[right] {$-$};
\draw[fill=white] (u2) circle (2pt) node[right] {$+$};


\draw (ky) -- ++ (0,-0.5) coordinate (u1);
\draw (ky2) -- ++ (0,-0.5) coordinate (u2);
\draw[fill=white] (u1) circle (2pt);
\draw[fill=white] (u2) circle (2pt); 
\end{circuitikz}
\end{document}